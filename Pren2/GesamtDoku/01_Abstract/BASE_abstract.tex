% !TEX root = ../Dokumentation.tex
\section*{Abstract}
\addcontentsline{toc}{section}{Abstract}
Die vorliegende Arbeit befasst sich mit der Umsetzung des Konzeptes für das Modell eines autonomen Müllfahrzeuges. Im Rahmen des Moduls \glqq{Produktentwicklung 2}\grqq an der Hochschule Luzern wird diese Umsetzung in einer interdisziplinären Zusammenarbeit zwischen Studierenden der Fachrichtungen Informatik, Elektrotechnik und Maschinenbau durchgeführt. Das umgesetzte Konzept wurde im Modul \glqq{Produktentwicklung 1}\grqq erarbeitet. Es wird aufgezeigt wie die Umsetzung eines Konzeptes durchgeführt wird, so das schlussendlich ein funktionierendes Fahrzeug vorgelegt werden kann. Der Fokus liegt auf dem Zusammenspiel von mechanischen, elektrischen und Software-Komponenten, der Herstellung der mechanischen Teile, dem Zusammenfügen aller Komponenten, Funkrtionstests und Inbetriebnahme des Fahrzeuges. Das Chassis wird mit vier Rädern realisiert. Die Lenkung ist eine von einem Servomotor angetriebene Achsschenkellenkung. Der Hauptantrieb erfolgt mittels einen DC-Motor mit Encoder. Für die Erzeugung der Bilder wird eine drehbare Raspberry Pi Cam verwendet. Die Bilder für die Fahrbahnerkennung werden mit OpenCV über ein Raspberry Pi 2 ausgewertet. Eine weitere Aufgabe des Raspberry Pi ist die Informationsverteilung. Für die Kommunikation mit den Hardwarekomponenten wird ein Mikrocontrollerboard von Freescale eingesetzt. Die genaue Detektierung des zu leerenden Containers erfolgt mittels Infrarotsensoren. Um den Container zu greifen wird ein Schwenkarm benutzt, welcher mit einem Geifer ausgestattet ist. Das Schüttgut wird über einen Behälter mit abgewinkelten Bodenflächen für das Entladen vorbereitet. Durch das Öffnen der Behälterklappe wird das Schüttgut in den Endbehälter entleert.
\clearpage