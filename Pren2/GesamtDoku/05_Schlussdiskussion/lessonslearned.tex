% !TEX root = Dokumentation.tex
\subsection{Lessons Learned}
Bei einer Projektarbeit mit interdisziplinärer Zusammenarbeit ist es wichtig, sich früh auf einen Konsens zu einigen.
Für das Team war es wichtig, sich immer gemeinsam zu entscheiden und so immer alle möglichen Auswirkungen im Blick zu haben.
Durch das Definieren von Schnittstellen zu Beginn, konnten viele Missverständnisse schon in der Planungsphase bereinigt werden.
\\[0.2cm]
Zu Beginn von Pren2 wurde gemeinsam die detaillierte Planung der Aufgaben und Meilensteine durchgeführt. Dies ermöglichte den Teammitgliedern Einsicht in die Abhängigkeiten zwischen den Disziplinen, welche es zu koordinieren galt. Doch auch trotz detaillierter Planug geriet das Team oft in Vorzug. Daher musste nach einigen Wochen der Terminplan noch einmal überarbeitet werden.
\\[0.2cm]
Um eine Gruppe zu bilden und erfolgreich zu führen braucht es zwingend zwei Dinge:
Die gleichen Ziele und die gleichen Regeln für alle Mitglieder.
Doch wo es Regeln gibt, wird es auch Verstösse geben. Deshalb hat sich das Team früh auf eine adäquate Strafe geeinigt.
Kommt jemand zu spät oder gar nicht, so soll derjenige das Team mit Getränken versorgen.
So ist sichergestellt, dass genügend Motivation vorhanden ist und auch die anderen bei solch einem Missgeschick profitieren.
\\[0.2]
Das Team ist stolz auf die geleistete Arbeit. Durch das Produktentwicklungs-Projekt konnten die Teammitglieder viele Erfahrungen sammeln, welche auch im beruflichen Alltag weiter bringe werden.