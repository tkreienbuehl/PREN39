% !TEX root = Dokumentation.tex
\subsection{Lessons Learned}
Bei einer Projektarbeit mit interdisziplinärer Zusammenarbeit ist es wichtig, sich früh auf einen Konsens zu einigen.
Für das Team war es wichtig, sich immer gemeinsam zu entscheiden und so immer alle möglichen Auswirkungen im Blick zu haben.
Durch das Definieren von Schnittstellen zu Beginn, konnten viele Missverständnisse schon in der Planungsphase bereinigt werden.
\\[0.2cm]
Gewisse OpenSource Lösungen bieten zwar eine breite und solide Palette von Funktionen an, sind aber schlecht auf eine spezifische Aufgabe skalierbar.
Ein interessiertes Mitglied der Gruppe hat sich die Zeit genommen und mit vertieftem mathematischen Wissen einen sehr effizienten Algorithmus entwickelt.
Dieser schlanke, auf die Aufgabe zugeschnittene, Algorithmus ermöglicht das Auswerten des Bildmaterials nahezu in Echtzeit.
\\[0.2cm]
Einige Mitglieder des Teams haben ein persönliches Engagement für Lego Bauteile. Dies machte sich nützlich, um ein erstes Fahrzeug zu visualisieren.
Mit diesem provisorischen Lego-Fahrzeug konnte die Strecke abgefahren werden. So wurden einige Problemstellen aufgedeckt, welche vorher nicht beachtet wurden. Durch diese simple Idee konnte die zukünftige Umsetzung massiv beeinflusst werden.
\\[0.2cm]
Zu Beginn setzte sich das Team zusammen und diskutierte über alle benötigten Funktionen und Bauteile.
Dies ermöglichte den Maschinenbauern ein Fahrzeug zu modellieren, welches genug Platz bietet und innerhalb der Grössenordnung bleibt.
Durch dieses Vorgehen ist frühzeitig sichergestellt, dass alle Funktionalitäten abgedeckt sind und zeitig Fertigungsteile bestellt werden können.
\\[0.2cm]
Um eine Gruppe zu bilden und erfolgreich zu führen braucht es zwingend zwei Dinge:
Die gleichen Ziele und die gleichen Regeln für alle Mitglieder.
Doch wo es Regeln gibt, wird es auch Verstösse geben. Deshalb hat sich das Team früh auf eine adäquate Strafe geeinigt.
Kommt jemand zu spät oder gar nicht, so soll derjenige das Team mit Getränken versorgen.
So ist sichergestellt, dass genügend Motivation vorhanden ist und auch die anderen bei solch einem Missgeschick profitieren.