% !TEX root = Dokumentation.tex
\subsection{Fehlerbehebung}
\subsubsection{Das Raspberry Pi ist nicht erreichbar}
\begin{itemize}
\item das Raspberry Pi ist nicht eingesteckt => einstecken
\item das Raspberry Pi läuft nicht => siehe Akkus
\item WLAN-Adapter nicht eingesteckt => einstecken
\item reboot
\item Netzwerk- und SSH-Konfiguration kontrollieren
\end{itemize}
\subsubsection{Nichts leuchtet}
\begin{itemize}
\item Die Akkus sind nicht verbunden => Verbinden
\item Alles ist korrekt verbunden => Überpfrüfe die Sicherungen, sind sie intakt?
\item Laufen die Bauteile einzeln => Verbinde das Raspberry und das Freedomboard einzeln mit dem PC und überprüfe die Funktionstüchtigkeit.
\end{itemize}

\subsubsection{Das Fahrzeug fährt nicht los}
\begin{itemize}
\item Blinkt ein LED auf dem Mikrocontrollerboard? => Einer der Akkus ist entladen(oder nicht angeschlossen) und muss geladen werden
\item Leuchtet das LED auf dem Motorentreiber nicht? => Der zweite Akku ist nicht verbunden oder ein Kabel ist nicht eingesteckt
\item Gibt das Freedomboard keine Antwort => Siehe Kapitel "Das Mikrocontrollerboard antwortet nicht"
\item Wurde kein StartFrd a gesendet => Sende StartFrd a damit das Fahrzeug Fahrbefehle entgegen nimmt
\item Wurde kein DCDr d \"\#geschwindigkeitswert\" gesendet => Sende z.B DCDr d 190 damit das Fahrzeug losfährt
\end{itemize}

\subsubsection{Das Mikrocontrollerboard antwortet nicht}
\begin{itemize}
\item Das Freedomboard und Raspberry Pi sind nicht verbunden =>verbinden
\item Die UART Einstellungen sind nicht =>Baudrate: 115200 Daten:8 Stop:1 Parity:None
\item Das Freedomboard leuchtet nicht => Freedomboard direkt mit dem PC verbinden und die ver

\end{itemize}
. 
\subsubsection{Ein Servo funktioniert nicht richtig}
Ist das Kabel richtig eingesteckt?
Sind beide Akkus verbunden? Es kann sein, dass gewisse Teile funktionieren ohne einen der beiden Akkus.
\subsubsection{Die Initialeinstellung stimmt nicht mehr}
\begin{itemize}
\item Hat sich eine Verschraubung gelöst? => Den Motor von Hand auf die Nullposition bringen und erneut befestigen.
\item Define-Werte im Code anpassen und rekalibrieren.

\end{itemize}
\subsubsection{Das Fahrzeug hält nach kurzer Zeit an}\\[0.2cm]
Dieser Fehler wird sehr wahrscheinlich 

\subsubsection{Erkennung funktioniert nicht richtig}
\begin{itemize}
\item Ist die Konfiguations-Datei am richtigen Ort?
\item Sind alle Konfigurationen richtig gesetzt?
\end{itemize}

