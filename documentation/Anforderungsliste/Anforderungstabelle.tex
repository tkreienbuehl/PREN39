% !TEX root = Anforderungsliste.tex
\begin{center}
\begin{tabular}{|p{1cm}|p{0.5cm}|p{5cm}|p{5cm}|p{1.5cm}|}\hline
\textbf{Nr.} & \textbf{F M W} & \textbf{Bezeichnung} & \textbf{Werte Daten Erläuterungen Änderungen} & \textbf{Ver- ant- wort- lich}\\\hline
\end{tabular}\\[0.3cm]
\begin{tabular}{|p{1cm}|p{0.5cm}|p{5cm}|p{5cm}|p{1.5cm}|}\hline
 \textbf{1} & & \textbf{Fahrgestell} & & \\\hline
 1.1 & F & Abmasse bie Fahrt und auf Parkplatz & 15cm Breit / 35cm Lang / 20cm Höhe (max Werte) & Alle\\ \hline
 1.2 & F & Automatischer Antrieb & & Alle\\\hline
 1.3 & F & Autonome Lenkung & & Alle\\\hline
 1.4 & F & Fahrzeug darf in der Kurve und beim Beladen nicht kippen & & Alle \\\hline
 1.5 & M & Lenkung & Kurvenradius 33cm & M \\\hline
 1.6 & W & Lenkung & Kurvenradius 30cm & M \\\hline
\end{tabular}\\[0.3cm]
\begin{tabular}{|p{1cm}|p{0.5cm}|p{5cm}|p{5cm}|p{1.5cm}|}\hline
 \textbf{2} & & \textbf{Fahren} & & \\\hline
 2.1 & F & Autonomes Abfahren der Strecke & & I\\\hline
 2.2 & F & Die Fahrbahn darf nicht verlassen werden & & Alle\\\hline
 2.3 & F & Es darf keine Kollisionen mit Fussgängern und Material auf dem Trotoir geben & & Alle \\\hline
 2.4 & F & Die Strecke muss innerhalb der vorgegebenen Zeit abgefahren werden & max. 4min & Alle\\\hline
 2.5 & W & Schnelleres Abfahren der Strecke & ca. 2.5min & Alle\\\hline
 2.6 & F & Positionsgenaues halten muss möglich sein & +/- 2cm & E / M\\\hline
 2.7 & F & Auf der Strecke gilt Rechtsvortritt & & I \\\hline
 2.8 & W & Nahes heranfahren ans Trottoir & bis au ca. 1cm & Alle\\\hline
 2.9 & F & Die Fahrt wird per Knopf gestartet & & I\\\hline
 2.10 & W & Die Fahrt wird per Fernsteuerung gestartet & & I \\\hline
\end{tabular}\\[0.3cm]
\begin{tabular}{|p{1cm}|p{0.5cm}|p{5cm}|p{5cm}|p{1.5cm}|}\hline
 \textbf{3} & & \textbf{Beladen} & & \\\hline
 3.1 & F & Autonomes Greifen des Containers & & M\\\hline
 3.2 & F & Beim beladen dürfen die Abmasse überschritten werden & & -\\\hline
 3.3 & F & Der Container muss vollständig in das Fahrzeug entleert werden & & M\\\hline
 3.4 & F & Nach der Leerung muss der Container wieder auf die Ausgangsüposition zurückgestellt werden & +/- 2cm & M\\\hline
 3.5 & F & Die gelben Container dürfen nicht entleert werden & & I\\\hline
 3.6 & F & Beim Greifen dürfen keine andern Container verschoben werden & & Alle\\\hline
 3.7 & F & Der Container muss einen genügenden Abstand zur Kreuzung haben, damit beim beladen die Kreuzung nicht blockiert wird & & - \\\hline   
\end{tabular}
\end{center}
