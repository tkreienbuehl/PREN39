% !TEX root = Konzeptvarianten.tex
\section{Boardcomputer}

Bei den Boardcomputern beschränkte sich die Recherche auf die Raspberry-Familie.
Diesem Mini-Computer geht ein gewisser Ruf voraus, der gewiss nicht unbegründet ist.
Ausserdem wurde er bisher in fast jedem PREN verwendet, da er für sehr viele Anwendungsbereiche eingesetzt werden kann.
Zusätzlich kennen sich einzelne Teammitglieder bereits ein wenig damit aus.
In die engere Auswahl kamen das Raspberry Pi (Pi), das Raspberry Pi 2 (Pi2) und das Banana Pi (BPi).


\begin{table}[h]
\begin{tabular}{|p{4.5cm}|p{3.5cm}|p{2cm}|p{2cm}|p{2cm}|}\hline
	
	\textbf{Kriterium}	& 	\textbf{Gewichtung (1-3)} & \textbf{Pi} & \textbf{Pi 2} & \textbf{BPi}\\\hline
	{Preis}	& 	{2} & {2} & {3} & {1}\\\hline
	{Grösse}	& 	{2} & {2} & {2} & {1}\\\hline
	{Leistung}	& 	{3} & {2} & {3} & {1}\\\hline
	{Ansteuerung}	& 	{2} & {3} & {3} & {2}\\\hline
	
	
	
	
\end{tabular}\\
\end{table}