% !TEX root = ../Dokumentation.tex
\section{Abstract}\\[0.2cm]
Die vorliegende Arbeit befasst sich mit der Konzeptfindung für ein Modell eines autonomen Müllfahrzeuges. Im Rahmen des Moduls Produktentwicklung 1 an der Hochschule Luzern, wird in einer in einer interdisziplinären Zusammenarbeit zwischen Studierenden der Fachrichtungen Informatik, Elektrotechnik und Maschinenbau dieses Konzept erarbeitet. Das Fahrzeug wird im Modul Produktentwicklung 2 hergestellt und muss eine vorgegebene Aufgabe ausführen können. Es wird aufgezeigt wie mehrere mögliche Konzepte entwickelt werden und schlussendlich ein Lösungskonzept zur Herstellung ausgewählt wird. Der Fokus liegt auf Projektmanagement, Budgetplanung, Technologierecherchen, Konzept Erarbeitung, interdisziplinärer Zusammenarbeit und Entscheidungsfindung. Anhand von Tests und Auswertungen wird das Chassis mit 4 Rädern, einer Servomotor angetriebenen Achsschenkellenkung und einem DC-Getriebemotor für den Antrieb bestimmt. Für die Erzeugung der Bilder wird eine drehbare Rasperry-Cam verwendet. Die Bilder für die Fahrbahnerkennung werden mit OpenCV über Raspberry-Pi 2 ausgewertet. Eine weitere Aufgabe des Rapberry-Pi, ist die Informationsverteilung. Für die Kommunikation mit den Hardwarekomponenten wird ein Microkontrollerboard von Freescale eingesetzt. Die genaue Detektierung des zu leerenden Containers erfolgt mittels Infrarotsensoren. Um den Container zu Greifen wird ein Schwenkarm benutzt, welcher mit einem Rundgreifer ausgestattet ist. Das Schüttgut wird über einen Behälter mit abgewinkelten Bodenflächen für das Entladen vorbereitet. Durch das öffnen der Behälterklappe wird das Schüttgut in den Endbehälter entleert.
\clearpage