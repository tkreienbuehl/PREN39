% !TEX root = ../Dokumentation.tex
\section*{Abstract}
\addcontentsline{toc}{section}{Abstract}
Die vorliegende Arbeit befasst sich mit der Konzeptfindung für das Modell eines autonomen Müllfahrzeuges. Im Rahmen des Moduls \glqq{Produktentwicklung 1}\grqq an der Hochschule Luzern wird in einer interdisziplinären Zusammenarbeit zwischen Studierenden der Fachrichtungen Informatik, Elektrotechnik und Maschinenbau dieses Konzept erarbeitet. Das Fahrzeug wird im Modul \glqq{Produktentwicklung 2}\grqq hergestellt und muss eine vorgegebene Aufgabe ausführen können. Es wird aufgezeigt wie mehrere mögliche Konzepte entwickelt werden und schlussendlich ein Lösungskonzept zur Herstellung ausgewählt wird. Der Fokus liegt auf Projektmanagement, Budgetplanung, Technologierecherchen, Konzepterarbeitung, interdisziplinärer Zusammenarbeit und Entscheidungsfindung. Anhand von Tests und Auswertungen wird das Chassis mit vier Rädern, von einem Servomotor angetriebenen Achsschenkellenkung und einem DC-Getriebemotor für den Antrieb bestimmt. Für die Erzeugung der Bilder wird eine drehbare Raspberry Pi Cam verwendet. Die Bilder für die Fahrbahnerkennung werden mit OpenCV über ein Raspberry Pi 2 ausgewertet. Eine weitere Aufgabe des Raspberry Pi ist die Informationsverteilung. Für die Kommunikation mit den Hardwarekomponenten wird ein Mikrocontrollerboard von Freescale eingesetzt. Die genaue Detektierung des zu leerenden Containers erfolgt mittels Infrarotsensoren. Um den Container zu greifen wird ein Schwenkarm benutzt, welcher mit einem Rundgreifer ausgestattet ist. Das Schüttgut wird über einen Behälter mit abgewinkelten Bodenflächen für das Entladen vorbereitet. Durch das Öffnen der Behälterklappe wird das Schüttgut in den Endbehälter entleert.
\clearpage