% !TEX root = Dokumentation.tex
\subsection{Lessons Learned}

Bei einer Projektarbeit mit interdisziplinärer Zusammenarbeit ist es wichtig sich früh auf einen Konsens zu einigen.
Für das Team war es wichtig sich immer gemeinsam zu entscheiden und so immer alle möglichen Auswirkungen im Blick zu haben.
Durch das Definieren von Schnittstellen bereits zu Begin, konnten viele Missverständnisse bereits in der Planungsphase bereinigt werden.

Gewisse OpenSource Lösungen bieten zwar eine breite und solide Palette von Funktionen an, sind aber schlecht auf eine spezifische Aufgabe skalierbar.
Ein begabteres Mitglied der Gruppe hat sich die Zeit genommen und mit vertieftem mathematischen Wissen einen sehr effizienten Algorithmus entwickelt.
Dieser schlanke, auf die Aufgabe zugeschnittene Algorithmus ermöglicht das Auswerten des Bildmaterials nahezu in Echtzeit.


Um eine Gruppe zu bilden und erfolgreich zu führen braucht es zwingend zwei Dinge.
Die gleichen Ziele und die gleichen Regeln für alle Mitglieder.
Doch wo es Regeln gibt, wird es auch Verstösse geben. Deshalb hat sich das Team früh auf eine adäquate Strafe geeinigt.
Kommt jemand zu spät oder gar nicht, so soll derjenige das nächste Mal Bier mitbringen.
So ist sichergestellt, dass genügend Motivation vorhanden ist und auch die anderen bei solch einem Missgeschick profitieren.