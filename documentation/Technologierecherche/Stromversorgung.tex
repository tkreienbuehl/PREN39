% !TEX root = Technologierecherche.tex
\section{Stromversorgung}

\subsection {Anforderungen}
\begin{itemize}
\item Muss die Energieversorgung gewährleisten können.
\item Darf sich aufgrund seiner Baugrösse und Gewichts nicht negativ auf das Fahrzeug auswirken
\end{itemize}

\subsection {Batterien}

\begin{itemize}
\item Einfache Beschaffung.
\item Günstig.
\item Relativ unempfindlich.
\item Platzsparende Montage möglich.
\item Einfache Energieregulierung durch Serie-, Parallelschaltung.
\item Geringe Selbstentladung.
\end{itemize}

\textbf {Nachteile}
\begin{itemize}
\item Nicht wieder aufladbar. 
\item Kurze Lebensdauer.	
\item Geringe Energiedichte.
\item Müssen regelmässig ersetzt werden.
\item Für den längeren und intensiven Gebrauch eher ungeeignet
\end{itemize}

\subsection {Lithium-Akku (LiPo, Lilon)}

\begin{itemize}
\item Grosse Auswahl von verschiedenen Bauformen.
\item Sehr hohe Energiedichte.
\item Geringe Selbstentladung.
\item Lange Lebensdauer.
\item Hoher Ladewirkungsgrad.
\item Relativ leicht.
\end{itemize}

\textbf {Nachteile}
\begin{itemize}
\item Empfindlich bei Über- und Unterschreiten der Spannungsgrenze.
\item Teuer.	
\item Benötigt teure Ladegeräte.	
\end{itemize}

\subsection {NiMH (Nickel-Metallhydrid)}
\begin{itemize}
\item Billig.
\item Lange Lebensdauer.
\item Relativ unempfindlich.
\item Simple und billige Ladegeräte.
\end{itemize}

\textbf {Nachteile}
\begin{itemize}
\item Relativ schwer.
\item Mittelmässige Energiedichte.
\item Mittelmässiger Ladewirkungsgrad.
\end{itemize}

