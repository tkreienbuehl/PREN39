% !TEX root = Technologierecherche.tex
\section{Spurerkennung}
Damit das Fahrzeug die Strecke autonom abfahren kann, muss das Fahrzeug in der Lage sein die Spur selbständig zu erkennen.
Die autonome Spurerkennung kann mit unterschiedlichen mitteln realisiert werden.

\subsection{Erkennung des rechten Randes mit Sensoren}
Der rechte Rand des Parcours ist für die Spurerkennung vielversprechend. Er ist entweder durch eine weisse Linie, ein 5mm hohes Trottoir oder für kurze Zeit auf der Kreuzung nicht begrenzt.
Für die unterschiedlichen Bedingungen werden verschieden Möglichkeiten aufgezeigt.\\

\textbf {Erkennung der weissen Linie}
Die rechte Begrenzungslinie ist 1cm breit und weiss. Es gibt viele Minaturfahrzeuge die einer Linie folgen. Jedoch ist bei den meisten Linienfolger ist die zu folgende Linie in der Mitte des Fahrzeuges. Das hat der Vorteil das man links und rechts der Linie die Linie detektieren kann. Dies ist bei diese Aufgabenstellung nicht oder nur sehr begrenzt möglich. Falls diese Variante gewählt würde wäre ein Funktionsmuster nötig, damit ....
Für die Linienerkennung gibt es bereits vorhandene Sensoren