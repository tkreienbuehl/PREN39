% !TEX root = Technologierecherche.tex
\section{Bilderkennung}

\subsection {Anforderungen}
\begin{itemize}
\item Objekterkennung (Container, Fahrzeug von rechts).
\item Linie erkennen.
\item Winkel der Linie erkennen.
\item Farbe erkennen.
\item Muss auf Linux oder Windows lauffähig sein.
\item Muss mit Java, C++ oder C-Sharp kompatibel sein.
\item Muss gut dokumentiert sein oder eine gute Community haben.
\end{itemize}

\subsection {OpenCV}

\begin{itemize}
\item Läuft auf den meisten Plattformen.
\item Stellt eine grosse Bibliotheke mit vielen Zusatzmodulen zur Verfügung.
\item Ist mit vielen grossen anderen Bibliotheken kompatibel.
\item Hat eine grosse Community.
\item Ist gratis und open source (BSD) lizenziert.
\item Ist gut dokumentiert.
\item Ist kompatibel mit Java, C, C++, Python usw..
\item Läuft sehr schnell.
\end{itemize}

\textbf {Nachteile}
\begin{itemize}
\item Enthält auch viele Funktionalitäten, welche nicht benötigt werden (komplex).		
\end{itemize}

\subsection {SimpleCV}

\begin{itemize}
\item Läuft auf den meisten Plattformen.
\item Ist mit OpenCV kompatibel.
\item Ist gratis und open source.
\item Hat eine grosse Community.
\item Ist gut dokumentiert.
\item Ist kompatibel mit Python.
\item Läuft schnell.
\item Leicht zu erlernen.
\end{itemize}
\textbf {Nachteile}
\begin{itemize}
\item Deckt möglicherweise nicht alle Anforderungen.	
\end{itemize}

\subsection {LibCCV}
\begin{itemize}
\item Läuft auf allen Plattformen.
\item Ist gratis und open source (BSD) lizenziert.
\end{itemize}
\textbf {Nachteile}
\begin{itemize}
\item Die Dokumentation ist mässig.
\end{itemize}




