% !TEX root = Technologierecherche.tex
\section{Recherchequellen}



\begin{table}[h]
\begin{tabular}{|p{3cm}|p{3.5cm}|p{5cm}|p{2cm}|}\hline
	
	\textbf{Themengebiet}	& 	\textbf{Beschreibung} & \textbf{Quelle} & \textbf{Bewertung (1-5)} \\\hline
	
	
	\textbf{Bilderkennung}	&	OpenCV Beschreibung	&	\url{http://docs.opencv.org/master/d1/dfb/intro.html#gsc.tab=0}	&	3 \\\hline
				 			&	Vergleich von OpenCV und SimpleCV	&	\url{http://simplecv.tumblr.com/post/19307835766/opencv-vs-matlab-vs-simplecv}	&	4 \\\hline
				 			&	SimpleCV Beschreibung	&	\url{http://simplecv.org/}	&	3 \\\hline
				 			&	Linienerkennung mit Open CV	&	\url{https://www.youtube.com/watch?v=aGGehlgiZoQ}	&	3	\\\hline
				 			
\textbf{Boardcomputer}	& 	Raspberry Pi & \url{https://www.pi-shop.ch/raspberry-pi-model-b} & 4 \\\hline
						& 	Raspberry Pi 2 & \url{https://www.pi-shop.ch/raspberry-pi-2-model-b} & 4 \\\hline
						& 	Banana Pi & \url{https://www.pi-shop.ch/banana-pi} & 4 \\\hline	
						
\textbf{Antrieb}	& 	Elektromotor & \url{http://www.modellbau-friedel.com} & 3 \\\hline
					& 	Verbrennungsmotor & \url{http://www.modellbau-friedel.com} & 3 \\\hline
					& 	Dampfmaschine & \url{www.modell-dampfmaschinen.de} & 3 \\\hline
					
\textbf{Lenkung} &  Beschreibung Knicklenkung & \url{http://www.portmanns.ch/Repetition/Fahrwerk/Lenkungsarten.pdf} & 4 \\\hline
 
	
\end{tabular}\\
\caption{Recherchequellen}
\end{table}