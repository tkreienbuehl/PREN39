% !TEX root = Technologierecherche.tex
\section{Kamerasysteme}
\subsection{Raspberry Pi CAM}
\begin{itemize}
\item Kosten: $\pm$30.-
\item Baugrösse: 25x20x9mm
\item Auflösung: 5 Megapixel, bis 30 Bilder pro Sekunde, 1080x720 HD
\item Schnittstelle: 15 Pin Flachband MIPI Kameraschnittstelle.
\item Lieferzeit: Innert 24h (Digitec)
\item Kamerawinkel Horizontal: 53.5 $\pm$ 0.13 Grad.
\item Kamerawinkel Vertikal: 41.41 $\pm$ 0.11 Grad.
\item Bildformate: JPEG (beschleunigt), JPEG + RAW, GIF, BMP, PNG, YUV420, RGB888
\item Brennweite: 3.6mm $\pm$ 0.01
\item Fixer Fokus: 1m bis unendlich
\item Software: keine, Kompatibel zu RaspberryPi 1 und 2
\end{itemize}
\subsection{Logitech Webcam C525}
\begin{itemize}
\item Kosten: $\pm$70.-
\item Baugrösse: 60x 40x 20mm geschätzt, keine genauen Angaben auf Herstellerseite.
\item Auflösung: 8 Megapixel, 1280x720 HD
\item Schnittstelle: USB 2.0
\item Lieferzeit: Innert 24h (Fust, Microspot, Logitech)
\item Kamerawinkel Horizontal: Keine Angaben
\item Kamerawinkel Vertikal: Keine Angaben
\item Bildformate: Keine Angaben
\item Brennweite: Keine Angaben
\item Fokus: Autofokus
\item Software: Logitech Software nur Windows Vista, 7 und 8.
\end{itemize}
\subsection{Logitech Webcam C615}
\begin{itemize}
\item Kosten: $\pm$100.-
\item Baugrösse: 80x 40x 20mm geschätzt, keine genauen Angaben auf Herstellerseite.
\item Auflösung: 8 Megapixel, 1920x1080 Full HD
\item Schnittstelle: USB 2.0
\item Lieferzeit: Innert 24h (Fust, Microspot, Logitech)
\item Kamerawinkel Horizontal: Keine Angaben
\item Kamerawinkel Vertikal: Keine Angaben
\item Bildformate: Keine Angaben
\item Brennweite: Keine Angaben
\item Fokus: Autofokus
\item Software: Logitech Software nur Windows Vista, 7 und 8.
\end{itemize}
\subsection{Logitech Webcam C920}
\begin{itemize}
\item Kosten: $\pm$100.-
\item Baugrösse: 80x 40x 20mm geschätzt, keine genauen Angaben auf Herstellerseite.
\item Auflösung: 5 Megapixel, 1080x720 HD
\item Schnittstelle: USB 2.0, VID\_046D{\&}PID\_0821
\item Lieferzeit: Innert 24h (Fust, Microspot, Logitech)
\item Kamerawinkel Diagonal: 83 Grad
\item Bildformate: Keine Angaben
\item Brennweite: 4.3mm
\item Fokus: Autofokus
\item Software: Logitech Software nur Windows Vista, 7, 8 und 10.
\end{itemize}
\subsection{Zusammenfassung}
Zusammenfassend können zwei Kamerasysteme verwendet werden. Dies sind einerseits die RaspberryPi Cam und andererseits eine Webcam, aus den Aufgeführten Modellen oder von weiteren Anbietern. Gemäss Tabelle xxx funktionieren diverse Modelle mit dem Raspberry Pi oder vergleichbaren Boards.
\subsection{Vor- und Nachteile}
\subsubsection{Raspberry Pi CAM}
\textbf{Vorteile:}
\begin{itemize}
\item Preis.
\item Open Source und sehr Detaillierte Funktionsangaben.
\item API vorhanden.
\item Baubrösse und Befestigung.
\item Schnittstelle auf Raspberry Pi abgestimmt.
\end{itemize}
\textbf{Nachteile:}
\begin{itemize}
\item Fixer Fokus ab 1m.
\item Winkel mit 53.5 Grad etwas klein.
\item Schnittstelle, wenn kein Raspberry Pi verwendet wird.
\end{itemize}
\subsubsection{Webcams}
\textbf{Vorteile:}
\begin{itemize}
\item Autofokus.
\item Hohe Auflösung.
\item Grössere Auswahl an Modellen.
\end{itemize}
\textbf{Nachteile:}
\begin{itemize}
\item Preis.
\item Baugrösse oft nicht genau angegeben.
\item Geringere Herstellerinformationen.
\item Fehlende API.
\item Befestigung, Gehäuse muss evtl entfernt werden.
\end{itemize}