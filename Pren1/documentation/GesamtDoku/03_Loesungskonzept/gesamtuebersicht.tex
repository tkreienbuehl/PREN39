% !TEX root = ../Dokumentation.tex
\subsection{Gesamtübersicht}
\begin{figure}[H]%Position festigen
\centering
\includegraphics[width=0.9\textwidth]{03_Loesungskonzept/pictures/uebersichtszeichnung.png}
\caption{Übersichtszeichnung}
\label{fig:Übersichtszeichnung}
\end{figure}
\subsubsection{Zusammenspiel}
Wie die obige Darstellung zeigt, besteht zwischen allen Komponenten ein enges Zusammenspiel. So wird beispielsweise die gesamte Bilderkennung über den Minicomputer gesteuert. Dieser erhält seine Daten durch die Kamera, aber auch von den Sensoren, welche über den Mikrocontroller an ihn gesendet werden. Nachdem die Informationen bearbeitet wurden, werden Befehle an den Mikrocontroller geschickt, welcher die Ansteuerung der Motoren übernimmt. Die Motoren wiederum, lösen die mechanischen Bewegungen aus, wie z.B. die Schwenkung der Kamera oder das Senken des Greifarmes.
\subsubsection{Schnittstellen}
\begin{figure}[H]%Position festigen
\centering
\includegraphics[width=0.9\textwidth]{03_Loesungskonzept/pictures/Verteilungsdiagramm.png}
\caption{Schnittstellenübersicht}
\label{fig:Verteilungsdiagramm}
\end{figure}
In Abbildung \ref{fig:Verteilungsdiagramm} wird eine detaillierte Ansicht der Komponente dargestellt, insbesondere was die Schnittstellen betrifft. So ist erkennbar, dass der Mikrocontroller als Hauptknoten fungiert und als Schnittstelle für analoge wie auch digitale Signale dient. Die Wahl der Schnittstellen ist bei fast allen Bauteilen vorgegeben. Zum Beispiel müssen die Ultraschallsensoren über I/O Pins angesteuert und ausgelesen werden. Anders sieht es bei der Verbindung Minicomputer und Mikrocontroller aus. Diese Schnittstelle war frei wählbar. Durchgesetzt hat sich UART\footnote{UART= Asynchrones serielles Kommunikationsprotokoll} über USB. Dies hat den Vorteil, das die Speisung und die Kommunikation über das selbe USB Kabel läuft. Zudem gibt es für UART bereits implementierte Funktionen auf Seiten- Mikrocontroller, wie auch Minicomputer.
\newpage
\subsubsection{Regelkreise}
\textbf{Geschwindigkeitsregelung}\\[0.2cm]
Damit das Fahrzeug mit einer konstanten Geschwindigkeit fahren kann, muss ein Regler eingebaut werden.
\begin{figure}[H]
	\centering
	\includegraphics[width=1\textwidth]{03_Loesungskonzept/pictures/Gesch_Regelung.png}
	\label{Regelung_Gesch}
	\caption{Grober Antriebsregelkreis}
\end{figure}
%In Abbildung \ref{Regelung_Gesch} 
Hier ist der grobe Regelkreis des Antriebs zu sehen. Der Sollwert wird vom Minicomputer vorgegeben und an das Mikrocontrollerboard weitergeleitet. In diesem wird dann die Regelung implementiert. Der Regelausgang ist ein PWM\footnote{PWM= Pulsweiten Moduliertes Signal}-Signal, welches die Geschwindigkeit des Antriebsmotors bestimmt. Die Drehzahl des Antriebmotors wird über einen Encoder zurück in den Mikrocontroller gelesen. Der Encoder misst die Drehzahl am Ausgang des DC-Antriebsmotors.\\[0.2cm]
\textbf{Lenkregelung}\\[0.2cm]
Damit des Fahrzeug auf der Strecke bleiben kann, braucht es einen Regelkreis für die Steuerung. Der grobe Regelkreis ist hier aufgezeigt.
\begin{figure}[H]
	\centering
	\includegraphics[width=1\textwidth]{03_Loesungskonzept/pictures/Lenk_Regelung.png}
	\label{Regelung_Lenken}
	\caption{Grober Steuerungs Regelkreis }
\end{figure}
%Wie in Abbildung \ref{Regelung_Lenken} 
Wie hier zu sehen ist, befindet sich der Regler auf dem Minicomputer. Der Regelausgang wird über UART auf das Mikrocontrollerboard weitergeleitet. Dies wird ein Winkel in Grad sein, um wie viel sich der Servo drehen soll. Der Mikrocontroller ändert daraufhin das PWM Signal zum Servo. Die Positionsdetektion wird über die Kamera/Flexsensor realisiert. Das ist nur eine Übersicht über den Regelkreis.
