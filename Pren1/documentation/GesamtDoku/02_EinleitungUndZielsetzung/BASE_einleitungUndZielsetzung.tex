% !TEX root = ../Dokumentation.tex
\section{Einleitung}
Busfahrer/in und Bus, Lokführer/in und Zug, Chauffeur/in und Auto, all diese Begriffe waren bis vor kurzer Zeit unmittelbar miteinander verbunden. Ohne den jeweils Anderen konnte man sich nicht fortbewegen. Doch genau dies ändert sich im Moment. Autonome Fahrzeuge sind längst nicht mehr nur wilde Fantasien, sondern werden von vielen Unternehmen entwickelt. Wenn man auf den Straßen von Kalifornien unterwegs ist, kann es bereits geschehen, dass ein solches autonom fahrendes Auto neben dem eigenen Wagen an der Ampel steht. Diese Entwicklung ist keineswegs neu, sondern hielt schleichend Einzug in die Automobil-Branche. Ob mit dem Tempomat oder den automatischen Notbremssystemen, die Grundbausteine wurden bereits in der Vergangenheit gelegt.
Trotzdem ist das autonome Fahren eine neue Dimension und erfordert sehr viel neue Technik. Das Potential und die Chancen für solche autonomen Fahrzeuge sind riesig. Jedoch sind auch die technischen Anforderungen sehr hoch, damit auch jeder Passagier sicher an sein Ziel kommt und alle Passanten sich angstfrei bewegen können. \\
Mit genau diesen Problemen ist auch die PREN Aufgabe 2015/2016 verbunden. Die vorliegende Arbeit soll aufzeigen, wie es möglich sein soll, ein autonomes Entsorgungsfahrzeug zu bauen. Dieses muss in der Lage sein auf einem achtförmigen Kurs selbständig zwei Miniatur-Abfallcontainer zu entleeren und das Schüttgut anschliessend auf einer Deponie zu entsorgen.\\
Für das optimale Lösungskonzept wurde die Aufgabenstellung in viele kleine Teilprobleme aufgeteilt. Zu jedem Teilproblem wurden verschiedene Lösungsvarianten gesucht. Die erfolgversprechendsten Varianten wurden näher betrachtet und teilweise mit Funktionsmuster ausgetestet. Diese Arbeit zeigt zum einen auf, welche Teilaufgaben bearbeitet und zum anderen, welche Lösungen ausgewählt wurden. Desweiteren sind die Projektstruktur und die Projektplanung aufgeführt. Diese Arbeit beschränkt sich auf die grundlegenden Konzepte. Detaillierte Konzepte oder Implementationen werden Bestandteil von PREN2 sein. 

\newpage