% !TEX root = morphkasten.tex

\section{Bilderkennungs-Bibliothek}


%##############
\subsection{OpenCV}

\begin{figure}[h!]%Position festigen
\centering
\includegraphics[width=0.3\textwidth]{fig/opencv.png}
\caption{OpenCV (Quelle: https://en.wikipedia.org/wiki/OpenCV)}
\label{fig:OpenCV}
\end{figure}

\begin{table}[h]
\begin{tabular}{p{0.5\textwidth} | p{0.5\textwidth}}



 \textbf{Vorteile} & \textbf{Nachteile} \\ \hline
	 
\begin{itemize}
\item Grosse Community
\item Läuft auf Linux
\item Kostenlos
\item Kompatibel mit Java und C++
\end{itemize}

 
 &
 
\begin{itemize}
\item Enthält auch viele Funktionen, welche nicht benötigt werden (komplex)
\end{itemize}

\end{tabular}
\end{table}

\begin{table}[h]
\begin{tabular}{p{0.5\textwidth}p{0.5\textwidth}}


 \textbf{Risiken} & \\ \hline
	 
\begin{itemize}
\item Ressourcenknappheit
\end{itemize}

 
\end{tabular}
\end{table}

\pagebreak


%##############
\subsection{SimpleCV}
\begin{figure}[h!]%Position festigen
\centering
\includegraphics[width=0.5\textwidth]{fig/simplecv.png}
\caption{SimpleCV (Quelle: http://google-opensource.blogspot.ch/2012\_08\_01\_archive.html)}
\label{fig:SimpleCV}
\end{figure}

\begin{table}[h]
\begin{tabular}{p{0.5\textwidth} | p{0.5\textwidth}}


 \textbf{Vorteile} & \textbf{Nachteile} \\ \hline
	 
\begin{itemize}
\item Läuft auf Linux
\item Freie Software
\item Kompatibel mit Java und C++
\end{itemize}

 &
 
\begin{itemize}
\item Community kleiner als bei OpenCV
\end{itemize}

\end{tabular}
\end{table}

\begin{table}[h]
\begin{tabular}{p{0.5\textwidth}p{0.5\textwidth}}


\textbf{Risiken} & \\ \hline
	 
\begin{itemize}
\item Zu Problemen wird keine Lösung gefunden
\item Ressourcenknappheit
\end{itemize}
 
\end{tabular}
\end{table}

\pagebreak